\documentclass[a4paper,aps,12pt,tightenlines]{revtex4-2}
\usepackage[a4paper,top=1.05in, bottom=0.75in, left=0.75in, right=0.75in]{geometry}
\usepackage[utf8]{inputenc}
\usepackage[T1]{fontenc}
\usepackage{microtype}
\usepackage[italian]{babel}
\makeatletter
\let\it@comma@def\active@comma
\makeatother
\usepackage{physics}
\usepackage{graphicx}
\usepackage{siunitx}
\usepackage[hidelinks]{hyperref}

\usepackage{pgf}
\usepackage{lmodern}

\DeclareMathOperator{\Corr}{Corr}
\DeclareMathOperator{\Cov}{Cov}
\DeclareMathOperator{\E}{E}
\DeclareMathOperator{\Var}{Var}

\begin{document}
\count\footins = 1000
\title{Misura delle righe di Balmer per H con metodi Monte Carlo}
\author{F. Polleri}
\email{s5025011@studenti.unige.it}
\author{M. Sotgia}
\email{s4942225@studenti.unige.it}
\affiliation{Dipartimento di Fisica, Università degli Studi di Genova, 16146 Genova, Italy}
\date{\today}
\maketitle

\section{Introduzione}
Sfruttando il fenomeno della rifrazione, un prisma ottico può essere utilizzato come spettrometro per ottenere misure precise della lunghezza d'onda di un fascio monocromatico incidente e anche per separare le componenti di un fascio non monocromatico. 

Si sa che infatti la differenza $\delta_m$ tra l'angolo di ingresso nel prisma $\theta_0$ e l'angolo d'uscita $\theta_m$ risulta essere legato al valore dell'indice di rifrazione del materiale, \begin{equation}\delta_m = \theta_0 - \alpha+\arcsin\left(n\sin\left(\alpha - \arcsin\left(\frac{\sin\theta_0}{n}\right)\right)\right),\end{equation} con $n$ indice di rifrazione e $\alpha$ apertura angolare del prisma. 

Si osserva che $\delta_m$ ha un minimo in corrispondenza del quale la misura è più stabile e la relazione precedente si semplifica come \begin{equation} n\sin\frac{\alpha}{2} = \sin\frac{\delta_m + \alpha}{2} \end{equation}

Da quest'ultima relazione possiamo ottenere una forma per l'indice \begin{equation} n(\theta_m, \theta_0) = \frac{\sin\frac{\theta_m-\theta_0 + \alpha}{2}}{\sin\frac{\alpha}{2}}.\label{eq:nthth0}\end{equation}

Però possiamo anche ricavare la relazione che lega la lunghezza d'onda $\lambda$ al valore di $\delta_m$ confrontando la relazione di Cauchy \begin{equation} n(\lambda) = A + \frac{B}{\lambda^2},\end{equation} appropriata ad un ordine $\mathcal O (1/\lambda^2)$, con $A$ e $B$ coefficienti propri del materiale in questione, e l'Eq. \eqref{eq:nthth0}

\subsection{Angoli di Balmer}

Dato un fascio di luce che attraversa un determinato materiale, si possono identificare le bande di assorbimento o di emissione; è stata definita da Balmer una relazione che permette di quantificare la posizione di queste bande \begin{equation} \frac{1}{\lambda} = R_H \qty(T(n) - T(m)) \end{equation} che nel caso dell'idrogeno assume una forma particolarmente comoda, per cui $T(n) = 1/n^2$. Otteniamo quindi un'equazione del tipo \begin{equation} \frac{1}{\lambda} = R_H \qty(\frac{1}{n^2} - \frac{1}{m^2}). \label{eq:ryd}\end{equation} Le lunghezze d'onda presenti nello spettro di emissione, o mancanti nel caso dello spettro di assorbimento, sono legate ognuna a degli angoli che possono essere determinati dallo strumento descritto sopra e appena caratterizzato. 

\section{Misura della costante di Rydberg}

Il prisma ottico, con le equazioni ad esso legate, insieme alla relazione di Balmer possono essere utilizzati per ricavare una stima del parametro $R_H$, detto costante di Rydberg. Sono noti gli angoli $\theta_m$ ai quali si trovano le bande di emissione dell'idrogeno, per $m = 3, 4, 5, 6$ e $n=2$, e si conoscono i parametri $A$, $B$ con le loro distribuzioni e il loro coefficiente di correlazione $\rho_{AB}$.

Possiamo allora determinare per ogni $\theta_m$ il valore $\lambda_m$ come \begin{equation} \frac{1}{\lambda_m} =  \sqrt{\frac{n(\theta_m, \theta_0) - A}{B}}, \label{eq:7}\end{equation} dove $n(\theta_m, \theta_0)$ è fornito dalla Eq. \eqref{eq:nthth0}. 
Possiamo quindi trovare una relazione tra i dati forniti e il valore di $R_H$. 
Come prima cosa possiamo calcolare il valore delle lunghezze d'onda a partire dai dati conosciuti. Conosciamo la forma effettiva di $\frac{1}{\lambda} = \frac{1}{\lambda}(\theta, \theta_0, A, B, \alpha)$, dove, a parte $\alpha$, i parametri presentano deviazione standard non trascurabile, comportando una correlazione tra i valori $\lambda_m$ al variare di $m$. Però, prima ancora di incontrare il problema legato alla correlazione tra le diverse lunghezze d'onda, dobbiamo capire come procedere per poter ottenere non solo il loro valore, ma anche il relativo errore. Dalle relazioni \eqref{eq:ryd} e \eqref{eq:7} osserviamo che la quantità che in realtà è utile ottenere è il fattore $1/\lambda$, comune alle due equazioni. 

Possiamo osservare che la dipendenza della lunghezza d'onda dai suoi parametri è molto probabilmente non lineare (si può vedere subito dalla presenza della radice quadrata che serve per ottenere il valore di $1/\lambda$), senza considerare che $A$ e $B$ sono tra loro legati da un coefficente di correlazione $\rho_{A,B}$. L'unico modo per ottenere il valore di $1/\lambda$ sarà quindi procedendo con metodi Monte Carlo (da qui in avanti MC). I parametri che presentano errore associato, e che non sono correlati tra loro e con le altre variabili ($\theta_m$, $\theta_0$), sono ipotizzati essere distribuiti secondo una Gaussiana, centrata nel calor medio $\mu_i$ e deviazione standard $\sigma_i$, con $i=\theta_m$, $\theta_0$. Il parametro $\alpha$ è considerato privo di errore. Supponendo di poter considerare $A$, $B$ come distribuzioni Gaussiane di valor medio $\mu_A$, $\mu_B$ e deviazione standard $\sigma_A$, $\sigma_B$ resta da capire come possiamo generare valori di $A$, $B$ che presentano invece coefficente di correlazione $\rho_{A,B} = \num{-0.872}$. Analizziamo in seguito il problema.

\subsection{Generazione di variabili correlate}
\emph{Il risultato che si ottiene è frutto di una ricerca e non tutto farina del nostro sacco, i crediti vanno a \cite{anthonyAnswerHowDoes2015, sobolevAnswerHowDoes2015, kaiserSamplePopulationScore1962}, anche se comunque siamo riusciti a rieseguire i calcoli e trovare la soluzione proposta. }

\begin{figure*}
\centering
\includegraphics[width=15cm]{../figures_and_tests/gaus_compAB.pdf}
\caption{Abbiamo generato $A$, $B$ prima utilizzando le formule, quindi a partire da $\lambda_i, \theta_i, \theta_0, \alpha$, poi utilizzando media e deviazione standard abbiamo ricostruito $A$ e $B$ come distribuzioni gaussiane, riportate in rosso.\label{fig:A_B_gaus}}
\end{figure*}

Sapendo che le distribuzioni di $A$, $B$ corrispondono a distribuzioni Gaussiane (fig. \ref{fig:A_B_gaus}), possiamo ipotizzare che siano legate tra loro imponendo che \begin{equation} \left\{ 
    \begin{aligned}
        & A = x_1 X _1 + x_2 X_2\\
        & B = X_1
    \end{aligned}
\right.\label{eq:8}
\end{equation}
con $x_1, x_2$ ignote e $X_1, X_2$ distribuzioni che consideriamo essere gaussiane. Sappiamo inoltre che \begin{equation} \Corr[A,B] = \rho = \frac{\Cov[A,B]}{\sqrt{\Var[A]\Var[B]}} =  \frac{x_1\Cov[X_1, X_1]}{\sqrt{\qty(x_1^2\Var[X_1] + x_2^2\Var[X_2])\Var[X_1]}} = \frac{x_1\sigma_1}{\sqrt{x_1^2\sigma_1^2+x_2^2\sigma_2^2}},\end{equation} oltre a sapere che \begin{equation} \rho = x_1\frac{\sigma_B}{\sigma_A} \implies x_1 = \rho\frac{\sigma_A}{\sigma_B}, \end{equation} dove abbiamo considerato che $\sigma_B = \sigma_1$ e $\mu_B = \mu_1$. 

Dalle relazioni in \eqref{eq:8} abbiamo ottenuto il sistema \begin{equation}\left\{
    \begin{aligned}
        \mu_A &= x_1 \mu_1 + x_2 \mu_2\\
        \mu_B &= \mu_1\\
        \sigma_A^2 &= x_1^2\sigma_1^2 + x_2^2\sigma_2^2\\
        \sigma_B^2 &= \sigma_1^2,
    \end{aligned}
\right.
\end{equation}
da cui abbiamo che, con le dovute sostituzioni, \begin{equation}\left\{
    \begin{aligned} 
    x_1 &= \rho \frac{\sigma_A}{\sigma_B}\\
    x_2 &= \sigma_A \sqrt{1-\rho^2}\\
    \mu_1 &= \mu_B\\
    \sigma_1 &= \sigma_B\\
    \mu_2 & = \left(\frac{\mu_A}{\sigma_A} - \rho\frac{\mu_B}{\sigma_B}\right)\frac{1}{\sqrt{1-\rho^2}}\\
    \sigma_2 &= 1
    \end{aligned}
\right.
\end{equation}

Abbiamo quindi la possibilità di ottenere in modo coerente con quanto atteso il valore di $A$ e il valore di $B$, generando quindi $B$ secondo una distribuzione Gaussiana $(\mu_B, \sigma_B)$ e calcolando invece $A$ come $A=\rho \frac{\sigma_A}{\sigma_B}\cdot B + \sigma_A \sqrt{1-\rho^2} \cdot X_2$, con $X_2$ generata come una Gaussiana \begin{equation}\left\{\begin{aligned}\mu_2 &= \left(\frac{\mu_A}{\sigma_A} - \rho\frac{\mu_B}{\sigma_B}\right)\frac{1}{\sqrt{1-\rho^2}}\\\sigma_2 &= 1\end{aligned}\right.\end{equation}.

\subsection{Valori di $R_H$ dagli angoli di Balmer}\label{sec:R_H}

Con metodi MC possiamo ora trovare i valori di $1/\lambda$, e successivamente, conoscendo la \eqref{eq:ryd}, potremmo realizzare un fit lineare per trovare il parametro $R_H$. Se considerassimo un fit minimizzando il valore del $\chi^2$ in maniera standard potremmo non considerare i coefficenti di correlazione che esistono dovuti al modo in cui stiamo calcolando i valori di $1/\lambda_m$, che dipendono tutti dagli stessi valori (con errore associato non trascurabile) di $A$, $B$ e $\theta_0$. 

Però possiamo calcolare per ogni coppia $(m, \lambda_m)$, o meglio $(m, 1/\lambda_m)$, un valore di $R_{H,m}$, come \begin{equation} 
    R_{H,m} = \frac{n^2m^2}{m^2-n^2}\sqrt{\frac{1}{B} \qty(\frac{\sin\frac{\theta_m-\theta_0 + \alpha}{2}}{\sin\frac{\alpha}{2}}-A)},
\end{equation} dove $\alpha$ è considerato privo di errore, $\theta_m, \theta_0$ sono considerati con errore, $n=2$, $m=3,4,5,6$ sono privi di errori, e $A$, $B$ sono generati come definito prima. Considerata questa relazione possiamo generare $N_\text{exp}$ esperimenti per ogni coppia $(m, \theta_m)$, per ognuno dei quali otteniamo il valore di $R_{H,m}$ per ogni valore di $m$, e possiamo così avere quattro distribuzioni per $R_H$ corrispondenti ai rispettivi valori di $m$ (come in fig. \ref{fig:rydberg}). 

\begin{figure}
\centering
\includegraphics[width=10.7cm]{../figures_and_tests/rydberg.pdf}
\caption{possiamo generare $N_\text{exp}$ esperimenti per ogni coppia $(m, \theta_m)$, per ognuno dei quali otteniamo il valore di $R_{H,m}$ per ogni valore di $m$, e possiamo così avere quattro distribuzioni per $R_H$ corrispondenti ai rispettivi valori di $m$. \label{fig:rydberg}}
\end{figure}

\subsection{Miglior stima del parametro $\hat R_H$}
Supponendo che le distribuzioni ottenute in \autoref{sec:R_H} siano distribuzioni gaussiane, che è una ipotesi valida considerato il teorema del limite centrale, conosciamo ora per ogni valore di $R_{H,m}$ una coppia $(\mu_m, \sigma_m)$, e conosciamo inoltre la forma numerica della matrice di correlazione, noti i coefficienti $\rho_{ij}$, dove $i,j=3,\dots, 6$, \begin{equation}
V_{ij}=\rho_{ij}\sigma_i\sigma_j=\left(
\begin{matrix}
\sigma_3^2 & \rho_{34}\sigma_3\sigma_4 & \rho_{35}\sigma_3\sigma_5 & \rho_{36}\sigma_3\sigma_6 \\
& \sigma_4^2 & \rho_{45}\sigma_4\sigma_5 & \rho_{46}\sigma_4\sigma_6 \\
& & \sigma_5^2 & \rho_{56}\sigma_5\sigma_6 \\
& & & \sigma_6^2 \\
\end{matrix}
\right).
\end{equation} Per trovare la miglior stima $\hat R_H$ possiamo sfruttare il metodo \emph{Generalized Least Squares} (GLS), e utilizzando uno stimatore costruito come \emph{Best Linear Unbiased} (BLU), possiamo ottenere che minimizzando la funzione definita come \begin{equation} S(\hat R_H) = \sum_{i,j = 3}^6 \qty(R_{H,i} - \hat R_H)\cdot V_{ij} \cdot \qty(R_{H,j} - \hat R_H) \end{equation} allora possiamo ottenere il valore della miglior stima del coefficente di Rydberg, avendo anche inoltre il valore della deviazione standrad associata alla misura. Con i dati forniti otteniamo il valore $\hat R_H = \SI{1.09601(107)e+07}{\per\meter}$.






\bibliography{corr_matrix_formula}

\end{document}
